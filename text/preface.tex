
\chapter*{Preface}\label{PREFACE}

\textit{Learn Prolog Now!} has a long and twisted history.  In 1995,
all three authors were based at the Department of Computational
Linguistics, at the University of the Saarland, in Saarbr\"{u}cken,
Germany. Johan, who was teaching the introduction to Prolog that year,
was working with Patrick on a Prolog-based introduction to natural
language semantics.%
\footnote{\textit{Representation and Inference for Natural Language: A
First Course in Computational Semantics}, Patrick Blackburn and Johan
Bos, CSLI Publications, 2005.}  
He decided to prepare a short set
of lecture notes on Prolog which could also be used as an Appendix to
the computational semantics book.

Nice idea, but that's not the way things worked out.  First, between
1996 and 2000, Patrick and Johan rethought the structure of the Prolog
courses, and along the way the notes became book-sized.  Then, from
2001 till 2004, Kristina took over the teaching, added new material
and (most importantly of all) turned \textit{Learn Prolog Now!}  into
a web-book.

It quickly became apparent that we had a hit on our hands: the website
got up to 4,000 visitors a month, and we received many
emails. Actually, this put us in a bit of a quandary. We wanted to
publish \textit{Learn Prolog Now!}  as a (low-budget) book --- but at
the same time we did \textit{not} want a publisher telling us that we
had to get rid of the free online version.

Luckily, Vincent Hendricks came to the rescue (thanks Vincent!).  He
told us about College Publications, Dov Gabbay's new publication
house, which was specifically designed to enable authors to retain
copyright. It was a marriage made in heaven.  Thanks to College
Publications we could make \textit{Learn Prolog Now!}  available in
book form at a reasonable price, and keep the web-book in place.

And that's the book you're now reading. It has been thoroughly tested,
first on nearly a decade's worth of students at Saarbr\"{u}cken, and
at the \textit{16th European Summer School in Logic, Language and
Information} which took place in Nancy, France, in August 2004, where
Kristina taught a hands-on introduction to Prolog. Though, as we hope
you will swiftly discover, you \textit{don't} need to be doing a
course to follow this book. We've tried to make \textit{Learn Prolog
Now!}  self-contained and easy to follow, so that it can be used
without a teacher. And as the feedback we have received confirms, this
is one of the most popular ways of using it.

So --- over to you. We had a lot of fun writing this. We hope you have
a lot of fun reading it, and that it really will help you to learn
Prolog now!

\subsection*{Acknowledgments}

Over the years that \textit{Learn Prolog Now!} existed as course notes
and web-book, we received many emails, ranging from helpful comments to
requests for answers to problems (a handful of which verged on demands
that we do their homework assignments!). We can't thank everyone by name,
but we did receive a lot of useful feedback this way and are very
grateful. And if we did any homework assignments, we ain't
telling\ldots

We are extremely grateful to Gertjan van Noord and Robbert Prins, who
used early versions of \textit{Learn Prolog Now!} in their teaching at
the University of Groningen. They gave us detailed feedback on its
weak points, and we've tried to take their advice into account; we
hope we've succeeded. We'd also like to say \textit{Grazie!} to
Malvina Nissim, who supplied us with an upgrade of
Exercise~\ref{L2.EX4}, helped format the final hardcopy version, and
generally gave us her enthusiastic support over many years.

Some special thanks are in order. First, we'd like to thank Dov Gabbay
for founding College Publications; may it do for academic publishing
what the GNU Public License did for software!  Second, heartfelt
thanks to Jane Spurr; we've \textit{never} had a more helpful,
competent, or enthusiastic editor, and \textit{nobody} reacts faster
than Jane. Thirdly, we like to thank Jan Wielemaker (the Linus
Torvalds of the Prolog world) for making SWI Prolog freely available
over the internet.  SWI Prolog is a an ISO-compliant Free Software
Prolog environment, licensed under the Lesser GNU Public License.  We
don't know what we'd have done without it. We're also very grateful to
him for the speedy and informative feedback he gave us on a number of
technical issues, and for encouraging us to go for ISO-standard
Prolog.  Finally, a big thank you to Ian Mackie and an anonymous
referee for all the time and energy they put into the penultimate version
of the book.

\begin{flushright}
Patrick Blackburn\\
Johan Bos\\
Kristina Striegnitz\\
\textit{May 2006}
\end{flushright}

