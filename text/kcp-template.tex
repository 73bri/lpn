\documentclass{book}

\usepackage{makeidx,named,amssymb,kcp,kcp-texts}
%\usepackage{jsinputs}
%\usepackage{url,itemlist,named}

%\usepackage{a4wide}

\input{pst-tree}
\input{pst-node}

\def\rmdefault{ptm}
\def\bfdefault{b}

\newcommand{\sfive}{{\bf S5}}
\renewcommand{\proves}{\mid\!\sim }
\makeindex

\newcommand\volume{5}
\newcommand\isbn{1-904987-00-1}
\newcommand\kcpAuthor{David Makinson}
\newcommand\kcpInst{King's College London}

\newcommand\TitleLineOne{B\sqz r\sqz i\sqz d\sqz g\sqz e\sqz s  f\sqz
r\sqz o\sqz m C\sqz l\sqz a\sqz s\sqz s\sqz i\sqz c\sqz a\sqz l}

\newcommand\TitleLineTwo{t\sqz o}

\newcommand\TitleLineThree{N\sqz  o\sqz  n\sqz  m\sqz  o\sqz  n\sqz
o\sqz t\sqz o\sqz n\sqz i\sqz c L\sqz o\sqz g\sqz i\sqz c}


\begin{document}
\thispagestyle{empty}

\kcpfrontmatter

\pagenumbering{roman}
\setcounter{page}{5}
\tableofcontents

\chapter*{Preface}
\addcontentsline{toc}{chapter}{Preface}
\markboth{PREFACE}{PREFACE}


\section*{Who are we writing for?}

This book is directed to all those who have heard of nonmonotonic
reasoning and would like to get a better idea of what it is all
about. What are its driving ideas? In what ways does it differ from
classical logic? How does it relate to probability? More generally,
how does it behave and how can it be used? We will try to answer these
questions as clearly as possible, without undue technicality on the
one hand, nor vagueness or hand waving on the other.

\section*{What does the reader need to know already?}

Tackling a subject such as this does require a minimal background, and
in honesty we must make it clear from the beginning what is needed. To
those without any grounding in classical logic at all, who may out of
curiosity have picked up the volume, we say: put it down, and go to
square one. Get the rudiments of classical propositional logic (alias
truth-functional logic) under your belt, and then come back. Otherwise
you will be like someone trying to learn about algebra before knowing
any arithmetic, counterpoint without sol-fa, or style without the
elements of grammar.



\section*{Main themes}

From the outside, nonmonotonic logic is often seen as a rather
mysterious affair. Even from the inside, it can appear to lack unity,
with multiple systems proposed by as many authors going in different
directions. The few available textbooks tend to perpetuate this
impression.





\section*{Topics Not Dealt With}


We owe it to the reader --- and especially to the instructor who may
be thinking of using the book --- to say clearly what topics,
associated with nonmonotonic reasoning, are \textit{not} covered.





\section*{Strategy of Presentation}

When the author first began planning this text, he followed engrained
habits and sought for maximal generality. `We have available a
multitude of different formal approaches to nonmonotonic reasoning',
he thought. `So in order to put some order into the affair, we need to
find the most general schema possible, under which they all fall. We
can then present the different accounts as so many special cases of
the general schema'.




\section*{Review and explore sections}

Each chapter ends with a `Review and explore' section. It contains
three packages to help the reader review the material covered and to
go further into the literature. First, a \textit{recapitulation} of
the essential message of the chapter. Second, a \textit{checklist} of
the key concepts, both formal and informal, that were introduced
there. The concepts are not explained all over again, but are simply
listed for checking off at review time. The reader can relocate the
formal definitions and informal explanations by looking for the
corresponding italicised terms in the main text. Third, a short
selection of texts for \textit{further reading}. Some of these cover
essentially the same ground as the section itself, albeit from a
different angle, in more detail or with particular attention to
aspects that we have not dwelt on. Some go further. In general, these
will be `bite size' --- short papers or chapters. Occasionally, a more
extended text is mentioned.


\section*{What is the best way to read this book?}

Do it pencil in hand; scribble in the margins. Take nothing on faith;
check out the assertions; find errors (and communicate them to the
author); pose questions.




\section*{For the instructor}

The author's experience is that in a graduate class, teaching can take
place at average of one section per hour, without counting section 1.1
and the brief `review and explore' sections. This adds up to about 20
hours of instruction time for the entire book, or about 15 hours for a
minimal version like the one mentioned above.




\section*{Conventions}

Theorems are numbered by their section, followed by a dash and a
digit. For example, the first explicitly displayed theorem in the book
occurs in section 1.3 and is numbered Theorem \ref{T1.3-1}. However,
only major results are displayed and numbered in this way. Many lesser
facts are simply stated in the flow of the text, and the attentive
reader should mark them as they are found. The same applies to
definitions. Only a few key ones are given special display; the others
are within the text, with their central terms italicized.




\section*{Acknowledgements}
\label{sec:acknowledgements}
\addcontentsline{toc}{section}{Acknowledgements}

\markboth{ACKNOWLEDGEMENTS}{ACKLNOWLEDGEMENTS}

As the basic ideas underlying this book were developed, they were
articulated in workshop and conference presentations, journal
publications, and the classroom.





\section*{Purpose}
\addcontentsline{toc}{section}{Purpose}
\markboth{PURPOSE}{PURPOSE}

This overview is directed to those who have heard of nonmonotonic
reasoning and would like to get a better idea of what it is all
about. What are its driving ideas? In what ways does it differ from
classical logic? How does it relate to probability? More generally,
how does it behave and how can it be used? We will try to answer these
questions as clearly as possible, without undue technicality on the
one hand, nor vagueness or hand waving on the other.
\cleardoublepage

\pagenumbering{arabic}
\setcounter{page}{0}

\chapter{Introduction}

\section{We Are All Nonmonotonic }


This overview is directed to those who have heard of nonmonotonic
reasoning and would like to get a better idea of what it is all
about. What are its driving ideas? In what ways does it differ from
classical logic? How does it relate to probability? More generally,
how does it behave and how can it be used? We will try to answer these
questions as clearly as possible, without undue technicality on the
one hand, nor vagueness or hand waving on the other.



This overview is directed to those who have heard of nonmonotonic
reasoning and would like to get a better idea of what it is all
about. What are its driving ideas? In what ways does it differ from
classical logic? How does it relate to probability? More generally,
how does it behave and how can it be used? We will try to answer these
questions as clearly as possible, without undue technicality on the
one hand, nor vagueness or hand waving on the other.



This overview is directed to those who have heard of nonmonotonic
reasoning and would like to get a better idea of what it is all
about. What are its driving ideas? In what ways does it differ from
classical logic? How does it relate to probability? More generally,
how does it behave and how can it be used? We will try to answer these
questions as clearly as possible, without undue technicality on the
one hand, nor vagueness or hand waving on the other.



This overview is directed to those who have heard of nonmonotonic
reasoning and would like to get a better idea of what it is all
about. What are its driving ideas? In what ways does it differ from
classical logic? How does it relate to probability? More generally,
how does it behave and how can it be used? We will try to answer these
questions as clearly as possible, without undue technicality on the
one hand, nor vagueness or hand waving on the other.



This overview is directed to those who have heard of nonmonotonic
reasoning and would like to get a better idea of what it is all
about. What are its driving ideas? In what ways does it differ from
classical logic? How does it relate to probability? More generally,
how does it behave and how can it be used? We will try to answer these
questions as clearly as possible, without undue technicality on the
one hand, nor vagueness or hand waving on the other.



This overview is directed to those who have heard of nonmonotonic
reasoning and would like to get a better idea of what it is all
about. What are its driving ideas? In what ways does it differ from
classical logic? How does it relate to probability? More generally,
how does it behave and how can it be used? We will try to answer these
questions as clearly as possible, without undue technicality on the
one hand, nor vagueness or hand waving on the other.



This overview is directed to those who have heard of nonmonotonic
reasoning and would like to get a better idea of what it is all
about. What are its driving ideas? In what ways does it differ from
classical logic? How does it relate to probability? More generally,
how does it behave and how can it be used? We will try to answer these
questions as clearly as possible, without undue technicality on the
one hand, nor vagueness or hand waving on the other.



This overview is directed to those who have heard of nonmonotonic
reasoning and would like to get a better idea of what it is all
about. What are its driving ideas? In what ways does it differ from
classical logic? How does it relate to probability? More generally,
how does it behave and how can it be used? We will try to answer these
questions as clearly as possible, without undue technicality on the
one hand, nor vagueness or hand waving on the other.



This overview is directed to those who have heard of nonmonotonic
reasoning and would like to get a better idea of what it is all
about. What are its driving ideas? In what ways does it differ from
classical logic? How does it relate to probability? More generally,
how does it behave and how can it be used? We will try to answer these
questions as clearly as possible, without undue technicality on the
one hand, nor vagueness or hand waving on the other.



This overview is directed to those who have heard of nonmonotonic
reasoning and would like to get a better idea of what it is all
about. What are its driving ideas? In what ways does it differ from
classical logic? How does it relate to probability? More generally,
how does it behave and how can it be used? We will try to answer these
questions as clearly as possible, without undue technicality on the
one hand, nor vagueness or hand waving on the other.



This overview is directed to those who have heard of nonmonotonic
reasoning and would like to get a better idea of what it is all
about. What are its driving ideas? In what ways does it differ from
classical logic? How does it relate to probability? More generally,
how does it behave and how can it be used? We will try to answer these
questions as clearly as possible, without undue technicality on the
one hand, nor vagueness or hand waving on the other.



This overview is directed to those who have heard of nonmonotonic
reasoning and would like to get a better idea of what it is all
about. What are its driving ideas? In what ways does it differ from
classical logic? How does it relate to probability? More generally,
how does it behave and how can it be used? We will try to answer these
questions as clearly as possible, without undue technicality on the
one hand, nor vagueness or hand waving on the other.



This overview is directed to those who have heard of nonmonotonic
reasoning and would like to get a better idea of what it is all
about. What are its driving ideas? In what ways does it differ from
classical logic? How does it relate to probability? More generally,
how does it behave and how can it be used? We will try to answer these
questions as clearly as possible, without undue technicality on the
one hand, nor vagueness or hand waving on the other.



This overview is directed to those who have heard of nonmonotonic
reasoning and would like to get a better idea of what it is all
about. What are its driving ideas? In what ways does it differ from
classical logic? How does it relate to probability? More generally,
how does it behave and how can it be used? We will try to answer these
questions as clearly as possible, without undue technicality on the
one hand, nor vagueness or hand waving on the other.



This overview is directed to those who have heard of nonmonotonic
reasoning and would like to get a better idea of what it is all
about. What are its driving ideas? In what ways does it differ from
classical logic? How does it relate to probability? More generally,
how does it behave and how can it be used? We will try to answer these
questions as clearly as possible, without undue technicality on the
one hand, nor vagueness or hand waving on the other.



This overview is directed to those who have heard of nonmonotonic
reasoning and would like to get a better idea of what it is all
about. What are its driving ideas? In what ways does it differ from
classical logic? How does it relate to probability? More generally,
how does it behave and how can it be used? We will try to answer these
questions as clearly as possible, without undue technicality on the
one hand, nor vagueness or hand waving on the other.



This overview is directed to those who have heard of nonmonotonic
reasoning and would like to get a better idea of what it is all
about. What are its driving ideas? In what ways does it differ from
classical logic? How does it relate to probability? More generally,
how does it behave and how can it be used? We will try to answer these
questions as clearly as possible, without undue technicality on the
one hand, nor vagueness or hand waving on the other.



This overview is directed to those who have heard of nonmonotonic
reasoning and would like to get a better idea of what it is all
about. What are its driving ideas? In what ways does it differ from
classical logic? How does it relate to probability? More generally,
how does it behave and how can it be used? We will try to answer these
questions as clearly as possible, without undue technicality on the
one hand, nor vagueness or hand waving on the other.



This overview is directed to those who have heard of nonmonotonic
reasoning and would like to get a better idea of what it is all
about. What are its driving ideas? In what ways does it differ from
classical logic? How does it relate to probability? More generally,
how does it behave and how can it be used? We will try to answer these
questions as clearly as possible, without undue technicality on the
one hand, nor vagueness or hand waving on the other.



This overview is directed to those who have heard of nonmonotonic
reasoning and would like to get a better idea of what it is all
about. What are its driving ideas? In what ways does it differ from
classical logic? How does it relate to probability? More generally,
how does it behave and how can it be used? We will try to answer these
questions as clearly as possible, without undue technicality on the
one hand, nor vagueness or hand waving on the other.



This overview is directed to those who have heard of nonmonotonic
reasoning and would like to get a better idea of what it is all
about. What are its driving ideas? In what ways does it differ from
classical logic? How does it relate to probability? More generally,
how does it behave and how can it be used? We will try to answer these
questions as clearly as possible, without undue technicality on the
one hand, nor vagueness or hand waving on the other.


\end{document}
